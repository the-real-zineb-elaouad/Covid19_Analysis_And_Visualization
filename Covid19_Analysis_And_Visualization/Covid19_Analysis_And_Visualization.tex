% Options for packages loaded elsewhere
\PassOptionsToPackage{unicode}{hyperref}
\PassOptionsToPackage{hyphens}{url}
%
\documentclass[
  11pt,
]{article}
\usepackage{amsmath,amssymb}
\usepackage{iftex}
\ifPDFTeX
  \usepackage[T1]{fontenc}
  \usepackage[utf8]{inputenc}
  \usepackage{textcomp} % provide euro and other symbols
\else % if luatex or xetex
  \usepackage{unicode-math} % this also loads fontspec
  \defaultfontfeatures{Scale=MatchLowercase}
  \defaultfontfeatures[\rmfamily]{Ligatures=TeX,Scale=1}
\fi
\usepackage{lmodern}
\ifPDFTeX\else
  % xetex/luatex font selection
\fi
% Use upquote if available, for straight quotes in verbatim environments
\IfFileExists{upquote.sty}{\usepackage{upquote}}{}
\IfFileExists{microtype.sty}{% use microtype if available
  \usepackage[]{microtype}
  \UseMicrotypeSet[protrusion]{basicmath} % disable protrusion for tt fonts
}{}
\makeatletter
\@ifundefined{KOMAClassName}{% if non-KOMA class
  \IfFileExists{parskip.sty}{%
    \usepackage{parskip}
  }{% else
    \setlength{\parindent}{0pt}
    \setlength{\parskip}{6pt plus 2pt minus 1pt}}
}{% if KOMA class
  \KOMAoptions{parskip=half}}
\makeatother
\usepackage{xcolor}
\usepackage[margin=1in]{geometry}
\usepackage{graphicx}
\makeatletter
\def\maxwidth{\ifdim\Gin@nat@width>\linewidth\linewidth\else\Gin@nat@width\fi}
\def\maxheight{\ifdim\Gin@nat@height>\textheight\textheight\else\Gin@nat@height\fi}
\makeatother
% Scale images if necessary, so that they will not overflow the page
% margins by default, and it is still possible to overwrite the defaults
% using explicit options in \includegraphics[width, height, ...]{}
\setkeys{Gin}{width=\maxwidth,height=\maxheight,keepaspectratio}
% Set default figure placement to htbp
\makeatletter
\def\fps@figure{htbp}
\makeatother
\setlength{\emergencystretch}{3em} % prevent overfull lines
\providecommand{\tightlist}{%
  \setlength{\itemsep}{0pt}\setlength{\parskip}{0pt}}
\setcounter{secnumdepth}{5}
\usepackage{tocloft}
\renewcommand{\cftdotsep}{1.5}
\renewcommand{\cftsecfont}{\normalfont}
\renewcommand{\cftsecpagefont}{\normalfont}
\ifLuaTeX
  \usepackage{selnolig}  % disable illegal ligatures
\fi
\IfFileExists{bookmark.sty}{\usepackage{bookmark}}{\usepackage{hyperref}}
\IfFileExists{xurl.sty}{\usepackage{xurl}}{} % add URL line breaks if available
\urlstyle{same}
\hypersetup{
  pdftitle={COVID-19 - Analyse et Visualisation},
  hidelinks,
  pdfcreator={LaTeX via pandoc}}

\title{\emph{COVID-19 - Analyse et Visualisation}}
\author{\emph{Réalisé par :}\\
EL AOUAD Zineb\\
DIALLO Mamadou Aliou\\
DIALLO Sadou\\
DOSSO Siaka\\
EL HIRI Salah Eddine}
\date{20 mai 2025}

\begin{document}
\maketitle

{
\setcounter{tocdepth}{2}
\tableofcontents
}
\hypertarget{introduction}{%
\section{Introduction}\label{introduction}}

Notre étude porte sur l'analyse comparative de l'évolution de la
COVID-19 des pays d'Europe les plus impactés, que sont la
\textbf{France}, l'\textbf{Allemagne}, le \textbf{Royaume-Uni},
l'\textbf{Italie} et la \textbf{Russie}.\\
L'observation est faite sur la période de \textbf{2020 à 2024}.

Dans le contexte de notre étude, nous avons procédé par \textbf{phases}
:

\begin{itemize}
\tightlist
\item
  La première phase a consisté au \textbf{traitement des données}.\\
\item
  La dernière phase a concerné \textbf{l'analyse}.
\end{itemize}

\hypertarget{muxe9thodologie}{%
\subsection{Méthodologie}\label{muxe9thodologie}}

L'analyse des données a mis en évidence : - l'\textbf{observation
globale} de l'évolution de la COVID-19 dans les pays du monde, - une
\textbf{proportionnalité par rapport au nombre d'habitants}, - puis une
\textbf{observation concentrée sur les cinq pays d'Europe} sélectionnés.

Pour mener à bien notre étude, nous avons divisé la tâche de travail en
différents \textbf{niveaux d'analyse} :

\begin{itemize}
\tightlist
\item
  \textbf{Une analyse d'évolution temporelle} portant sur l'évolution du
  nombre de cas de COVID détectés.
\item
  \textbf{Une analyse comparative} permettant de mettre en évidence :

  \begin{itemize}
  \tightlist
  \item
    la différence du nombre de cas détectés entre les pays,
  \item
    le \textbf{taux linéaire} d'évolution.
  \end{itemize}
\item
  \textbf{Des indices d'indicateurs} pour illustrer l'impact des
  \textbf{mesures anti-COVID-19} sur l'évolution du virus, en
  corrélation avec le \textbf{niveau de rigueur} propre à chaque pays.
\item
  \textbf{Une analyse post-COVID-19}.
\end{itemize}

\hypertarget{source-des-donnuxe9es}{%
\subsection{Source des données}\label{source-des-donnuxe9es}}

Les données utilisées proviennent du dépôt GitHub de \emph{Our World In
Data}, accessible à l'adresse suivante :\\
\url{https://github.com/owid/covid-19-data/tree/master/public/data}\strut \\
Le fichier README de ce dépôt fournit des précisions sur les sources et
la structure des données.

\hypertarget{sources-dinformation}{%
\subsection{Sources d'information}\label{sources-dinformation}}

\begin{itemize}
\tightlist
\item
  \emph{Our World In Data} :
  \url{https://ourworldindata.org/coronavirus}\\
\item
  L'article dédié aux cas de COVID-19 :
  \url{https://ourworldindata.org/covid-cases} d'OWID
\item
  \textbf{Google} : utilisé pour la vérification des données et des
  tendances réelles, en complément de nos résultats\\
\item
  \textbf{ChatGPT} : pour la validation des informations ainsi que pour
  la résolution d'erreurs et de bugs dans le code, proposition de graphe
  adapté à une étude
\end{itemize}

\hypertarget{packages-uxe0-installer-uxe0-duxe9commenter-au-besoins}{%
\subsection{Packages à installer (à décommenter au besoins)
:}\label{packages-uxe0-installer-uxe0-duxe9commenter-au-besoins}}

\hypertarget{chargement-exploration-pruxe9paration-et-nettoyage-des-donnuxe9es}{%
\section{Chargement, Exploration, Préparation et nettoyage des
données}\label{chargement-exploration-pruxe9paration-et-nettoyage-des-donnuxe9es}}

\hypertarget{chargement}{%
\subsection{Chargement}\label{chargement}}

\hypertarget{vuxe9rification-et-convertions-des-types-de-variables}{%
\subsection{Vérification et convertions des types de
variables}\label{vuxe9rification-et-convertions-des-types-de-variables}}

\hypertarget{conversion-de-la-date-et-des-variables-catuxe9gorielles}{%
\subsubsection{Conversion de la date et des variables
catégorielles}\label{conversion-de-la-date-et-des-variables-catuxe9gorielles}}

\hypertarget{suppression-des-colonnes-avec-plus-de-90-de-valeurs-manquantes}{%
\subsection{Suppression des colonnes avec plus de 90 \% de valeurs
manquantes}\label{suppression-des-colonnes-avec-plus-de-90-de-valeurs-manquantes}}

toutes les colonnes contenant plus de 90\% de valeurs manquantes (NA)
ont été retirées. Leurs noms sont conservés dans une variable
colonnes\_supprimees à des fins de traçabilité.

\hypertarget{colonnes-supprimuxe9s}{%
\subsubsection{Colonnes Supprimés :}\label{colonnes-supprimuxe9s}}

\begin{verbatim}
##  [1] "icu_patients"                           
##  [2] "icu_patients_per_million"               
##  [3] "hosp_patients"                          
##  [4] "hosp_patients_per_million"              
##  [5] "weekly_icu_admissions"                  
##  [6] "weekly_icu_admissions_per_million"      
##  [7] "weekly_hosp_admissions"                 
##  [8] "weekly_hosp_admissions_per_million"     
##  [9] "excess_mortality_cumulative_absolute"   
## [10] "excess_mortality_cumulative"            
## [11] "excess_mortality"                       
## [12] "excess_mortality_cumulative_per_million"
\end{verbatim}

\hypertarget{duxe9tection-traitement-et-suppression-des-doublons}{%
\subsection{Détection, traitement et suppression des
doublons}\label{duxe9tection-traitement-et-suppression-des-doublons}}

\hypertarget{detectetion}{%
\subsubsection{- Detectetion}\label{detectetion}}

\hypertarget{nous-avons-duxe9tectuxe9-7-770-doublons-sur-la-combinaison-location-date.-ces-doublons-nuxe9taient-pas-de-simples-ruxe9puxe9titions-mais-souvent-deux-lignes-contenant-des-donnuxe9es-partielles.-cest-des-donnees-fragmentees}{%
\paragraph{Nous avons détecté 7 770 doublons sur la combinaison location
+ date. Ces doublons n'étaient pas de simples répétitions, mais souvent
deux lignes contenant des données partielles. C'EST DES DONNEES
FRAGMENTEES}\label{nous-avons-duxe9tectuxe9-7-770-doublons-sur-la-combinaison-location-date.-ces-doublons-nuxe9taient-pas-de-simples-ruxe9puxe9titions-mais-souvent-deux-lignes-contenant-des-donnuxe9es-partielles.-cest-des-donnees-fragmentees}}

\hypertarget{traitement-des-doublons}{%
\subsubsection{Traitement des doublons}\label{traitement-des-doublons}}

\hypertarget{on-a-fusionnuxe9-chaque-groupe-de-doublons-location-date-en-une-seule-ligne-en-conservant-pour-chaque-colonne-la-valeur-non-manquante-non-na-si-elle-existe.-si-les-deux-valeurs-sont-na-le-na-est-conservuxe9}{%
\paragraph{On a fusionné chaque groupe de doublons (location, date) en
une seule ligne, en conservant Pour chaque colonne, la valeur non
manquante (non-NA) si elle existe. Si les deux valeurs sont NA, le NA
est
conservé}\label{on-a-fusionnuxe9-chaque-groupe-de-doublons-location-date-en-une-seule-ligne-en-conservant-pour-chaque-colonne-la-valeur-non-manquante-non-na-si-elle-existe.-si-les-deux-valeurs-sont-na-le-na-est-conservuxe9}}

\hypertarget{suppression-des-doublons-apruxe8s-fusion}{%
\subsubsection{Suppression des doublons après
fusion}\label{suppression-des-doublons-apruxe8s-fusion}}

\hypertarget{suppression-des-anciennes-lignes-doubluxe9es-du-jeu-de-donnuxe9es-initial-ajout-des-lignes-fusionnuxe9es-uxe0-leur-place}{%
\paragraph{Suppression des anciennes lignes doublées du jeu de données
initial + Ajout des lignes fusionnées à leur
place}\label{suppression-des-anciennes-lignes-doubluxe9es-du-jeu-de-donnuxe9es-initial-ajout-des-lignes-fusionnuxe9es-uxe0-leur-place}}

\hypertarget{revuxe9rification-des-donees-et-chargement-dans-coviddata_prepared-variable-avec-laquelle-on-va-travailler-tout-le-long}{%
\subsection{ReVérification des donees et Chargement dans
CovidData\_Prepared (variable avec laquelle on va travailler tout le
long)}\label{revuxe9rification-des-donees-et-chargement-dans-coviddata_prepared-variable-avec-laquelle-on-va-travailler-tout-le-long}}

\hypertarget{ajout-dune-colonne-type_location-qui-indique-selon-une-location-si-cest-un-pays-un-continent-un-territoires-ou-duxe9pendances-une-aggregation-total-location-255}{%
\subsection{Ajout d'une colonne type\_location qui indique selon une
location si c'est un Pays, un Continent, un territoires ou dépendances,
une aggregation -- Total location =
255}\label{ajout-dune-colonne-type_location-qui-indique-selon-une-location-si-cest-un-pays-un-continent-un-territoires-ou-duxe9pendances-une-aggregation-total-location-255}}

\hypertarget{analyses-des-donnuxe9es}{%
\section{Analyses des données}\label{analyses-des-donnuxe9es}}

\hypertarget{nombre-total-de-cas-covid-19-par-pays-sur-la-carte}{%
\subsubsection{Nombre total de cas COVID-19 par pays sur la
carte}\label{nombre-total-de-cas-covid-19-par-pays-sur-la-carte}}

\includegraphics{Covid19_Analysis_And_Visualization_files/figure-latex/unnamed-chunk-19-1.pdf}

\hypertarget{interpretention}{%
\paragraph{Interpretention :}\label{interpretention}}

\textbf{Le Graphe}\\
Ce graphique montre la répartition mondiale des cas cumulés de COVID-19,
avec un code couleur allant du violet (peu de cas) au jaune vif (très
grand nombre de cas, jusqu'à 100 millions).

\begin{itemize}
\item
  \textbf{En jaune} : les pays ayant enregistré le plus grand nombre de
  cas cumulés (ex : États-Unis, Inde, Brésil, France).
\item
  \textbf{En orange/rouge} : pays avec plusieurs millions de cas (ex :
  Canada, Espagne, Afrique du Sud, Russie).
\item
  \textbf{En violet foncé} : pays ayant relativement peu de cas déclarés
  (ex : beaucoup de pays d'Afrique centrale, quelques pays d'Asie du
  Sud-Est).

  \textbf{Interprétation concrète}\\
  Les pays les plus industrialisés et les plus peuplés présentent
  souvent les nombres de cas les plus élevés, mais disposent également
  d'une capacité de dépistage plus importante.
\end{itemize}

L'Afrique apparaît majoritairement en violet ou rouge foncé : cela peut
traduire un nombre réel de cas plus faible, ou un sous-dépistage (moins
de tests, moins de données disponibles).

Certains pays, comme la Chine ou la Corée du Nord, peuvent sembler moins
touchés, mais cela peut aussi résulter des politiques de déclaration ou
d'un manque de transparence dans les données.

Le graphique ne reflète pas uniquement la gravité de la situation
sanitaire, mais aussi les capacités de test, les politiques de santé
publique et la démographie. Il permet de visualiser rapidement les zones
les plus touchées en nombre absolu.

\hypertarget{uxe9volution-mondiale-des-cas-et-des-duxe9cuxe8s-de-covid-19}{%
\subsubsection{Évolution Mondiale des Cas et des Décès de
COVID-19}\label{uxe9volution-mondiale-des-cas-et-des-duxe9cuxe8s-de-covid-19}}

\includegraphics{Covid19_Analysis_And_Visualization_files/figure-latex/unnamed-chunk-20-1.pdf}

\hypertarget{analyse-comparative-de-levolution-de-la-covid-19-dans-les-5-pays-deurope-les-plus-impactuxe9s}{%
\subsection{Analyse comparative de l'evolution de la Covid 19 dans les 5
pays d'Europe les plus
impactés}\label{analyse-comparative-de-levolution-de-la-covid-19-dans-les-5-pays-deurope-les-plus-impactuxe9s}}

\hypertarget{ruxe9cupuxe9rerer-les-5-pays-deurope-les-plus-touchuxe9es-par-le-covid-19-en-se-basant-sur-le-nombre-total-de-cas-enregister-pour-chaque-pays}{%
\subsubsection{récupérerer les 5 pays d'europe les plus touchées par le
covid-19 en se basant sur le nombre total de cas enregister pour chaque
pays}\label{ruxe9cupuxe9rerer-les-5-pays-deurope-les-plus-touchuxe9es-par-le-covid-19-en-se-basant-sur-le-nombre-total-de-cas-enregister-pour-chaque-pays}}

\hypertarget{le-graphe-repruxe9sentant-nombre-moyen-par-jour-de-nouveaux-cas-enregistuxe9-pour-les-5-pays-les-plus-touchuxe9s-deurope}{%
\subsubsection{Le graphe représentant nombre moyen par jour de nouveaux
cas enregisté pour les 5 pays les plus touchés
d'Europe}\label{le-graphe-repruxe9sentant-nombre-moyen-par-jour-de-nouveaux-cas-enregistuxe9-pour-les-5-pays-les-plus-touchuxe9s-deurope}}

\hypertarget{uxe9volution-de-la-panduxe9mie-dans-cinq-pays-europuxe9ens}{%
\paragraph{Évolution de la pandémie dans cinq pays
européens}\label{uxe9volution-de-la-panduxe9mie-dans-cinq-pays-europuxe9ens}}

Nous observons l'évolution de la pandémie pour chacun des cinq pays,
indépendamment les uns des autres.

Les cinq pays d'Europe les plus touchés par la pandémie sont : -
\textbf{France} : 20 \% du taux de contamination en Europe -
\textbf{Allemagne} : 19 \% - \textbf{Italie} : 13 \% -
\textbf{Royaume-Uni} : 12 \% - \textbf{Russie} : 10 \%

Le taux global de contamination dans ces pays représente environ
\textbf{72 \%} de l'ensemble des cas détectés en Europe.

\begin{center}\rule{0.5\linewidth}{0.5pt}\end{center}

\begin{itemize}
\item
  Analyse par pays
\item
  • France En début 2020, on observe une tendance moyennement importante
  du nombre de cas détectés, qui prend une proportion non négligeable en
  octobre 2020. Cette tendance reste constante jusqu'à octobre 2021 où
  l'on observe une croissance rapide du nombre de cas.
\end{itemize}

Le nombre de nouveaux cas quotidiens a atteint son pic fin 2021 et début
2022, frôlant les \textbf{350 000 cas par jour}, dépassant largement les
vagues précédentes. Durant le deuxième trimestre de l'année 2022, une
chute importante du nombre de cas est observée. Depuis, la situation
s'améliore progressivement, bien qu'on note encore quelques petites
vagues entre la fin du deuxième trimestre 2022 et 2023. À la fin de
2023, le nombre de cas s'est stabilisé à un niveau assez bas, jusqu'à
devenir \textbf{négligeable début 2024}.

\begin{itemize}
\item
  • Allemagne L'Allemagne a également connu une forte vague début 2022,
  avec un pic autour de \textbf{200 000 cas par jour}. Par la suite,
  plusieurs vagues, moins importantes, se sont succédé. La courbe des
  contaminations est restée fluctuante, mais la \textbf{tendance
  générale est à la baisse à partir de 2023}.
\item
  • Italie En début 2020, une tendance modérément importante du nombre
  de cas détectés est observée, qui augmente significativement en
  octobre 2020. Cette tendance reste stable jusqu'à octobre 2021, moment
  où la croissance des cas s'accélère.
\end{itemize}

Ce schéma ressemble à celui de la France, avec un pic important début
2022, atteignant près de \textbf{180 000 cas quotidiens}. Entre 2020 et
2023, plusieurs vagues se sont succédé, atteignant \textbf{100 000 cas
entre juin et août 2022}, mais leur intensité a diminué avec le temps,
montrant un \textbf{meilleur contrôle de la situation}.

\begin{itemize}
\item
  • Royaume-Uni Un pic très élevé a été enregistré début 2022, avec près
  de \textbf{200 000 cas quotidiens}. Ensuite, les cas chutent
  rapidement, avec seulement quelques petites vagues par la suite. Dès
  2023, la courbe diminue fortement et les nouveaux cas deviennent
  \textbf{quasiment inexistants en 2024}, signe d'une sortie progressive
  de la crise.
\item
  • Russie Les pics épidémiques sont plus étalés et \textbf{moins
  élevés} qu'en France ou en Allemagne. Le pic le plus marqué atteint
  environ \textbf{150 000 cas par jour}. Après la mi-2023, la courbe
  devient presque plate, ce qui pourrait s'expliquer par une
  \textbf{baisse réelle de la circulation du virus} ou une
  \textbf{modification dans la manière de tester et de rapporter les
  cas}.
\end{itemize}

\hypertarget{le-graphe-repruxe9sentant-nombre-moyen-par-jour-de-nouveaux-duxe9cuxe9s-enregistuxe9-pour-les-5-pays-les-plus-touchuxe9s-deurope}{%
\subsubsection{Le graphe représentant nombre moyen par jour de nouveaux
décés enregisté pour les 5 pays les plus touchés
d'Europe}\label{le-graphe-repruxe9sentant-nombre-moyen-par-jour-de-nouveaux-duxe9cuxe9s-enregistuxe9-pour-les-5-pays-les-plus-touchuxe9s-deurope}}

\hypertarget{interpretation}{%
\paragraph{Interpretation}\label{interpretation}}

🇫🇷 \textbf{France}\\
En France, trois grandes vagues de décès se distinguent. La première a
lieu entre mi-avril et mi-mai, avec un pic proche de 500 morts par jour.
La deuxième, survenue à la fin de l'année 2020, est la plus importante,
atteignant environ 800 décès quotidiens. La troisième, début 2022,
avoisine les 300 morts par jour, malgré une nette baisse observée les
mois précédents. Par la suite, chaque nouvelle vague de COVID-19
entraîne une hausse de mortalité moins marquée.

🇩🇪 \textbf{Allemagne}\\
L'évolution en Allemagne est similaire à celle observée en France, avec
un pic notable fin décembre 2021, autour de 900 décès par jour. Les
vagues suivantes sont plus fréquentes mais moins intenses, avec une
mortalité qui s'étale davantage dans le temps. À partir de la mi-2022,
les augmentations deviennent moins fortes, bien que la courbe reste plus
irrégulière et prolongée que dans d'autres pays.

🇮🇹 \textbf{Italie}\\
L'Italie est l'un des premiers pays touchés, avec une première vague
particulièrement meurtrière dès mars-avril 2020, atteignant près de 800
morts par jour. Une nouvelle vague significative apparaît en décembre
2020. Ensuite, la mortalité diminue, même si les vagues continuent de se
succéder.

🇷🇺 \textbf{Russie}\\
La Russie se distingue par des pics de mortalité très élevés et plus
durables que dans les autres pays. Le pic maximal est enregistré en
novembre 2021, avec plus de 1 200 morts par jour. Contrairement à
l'Europe de l'Ouest, la Russie reste confrontée à une mortalité élevée
sur une période prolongée.

🇬🇧 \textbf{Royaume-Uni}\\
Le Royaume-Uni connaît deux pics majeurs de décès : le premier en
janvier 2021, avec près de 1 400 morts par jour --- un record parmi les
cinq pays analysés ---, et le second fin avril 2020, avec environ 1 200
décès quotidiens. Après ces vagues, la mortalité chute rapidement. Même
lors des vagues de contaminations en 2022, le nombre de décès reste
relativement faible.

\textbf{Synthèse comparative}\\
La comparaison entre les cinq pays montre que la mortalité liée à la
COVID-19 a culminé entre fin 2020 et début 2021, avant la généralisation
des campagnes de vaccination. Le Royaume-Uni et la Russie présentent des
pics particulièrement élevés, tandis que l'Italie se distingue par une
première vague très précoce dès mars 2020. À partir de 2022, la tendance
générale est à la baisse de la mortalité dans tous les pays.

\hypertarget{evolution-des-personnes-ayant-reuxe7u-au-moins-une-dose-de-vaccin}{%
\subsubsection{Evolution des personnes ayant reçu au moins une dose de
vaccin}\label{evolution-des-personnes-ayant-reuxe7u-au-moins-une-dose-de-vaccin}}

\includegraphics{Covid19_Analysis_And_Visualization_files/figure-latex/unnamed-chunk-24-1.pdf}

\hypertarget{interpretation-1}{%
\paragraph{Interpretation}\label{interpretation-1}}

Les estimations de la population que nous utilisons pour calculer les
paramètres de mesure par habitant sont fondées sur la dernière révision
des \textbf{Perspectives démographiques mondiales} des Nations Unies.

La vaccination a été réalisée dans tous les pays, impactés ou non par la
pandémie, et est devenue une obligation nécessaire pour les conditions
de déplacement d'un pays à un autre sur la période de \textbf{2020 à
2024}.

L'objectif de cette étude est d'analyser la variation des doses
administrées et la masse de population ayant reçu au moins une dose de
vaccin, ainsi que l'impact que ces mesures ont eu sur la propagation de
la COVID-19.\\
\textbf{Est-elle un facteur d'évolution ou un frein ?}

Pour cela, nous allons observer le graphe d'évolution du taux
d'injection en parallèle du graphe d'évolution de la pandémie.

\begin{itemize}
\tightlist
\item
  \textbf{1,3 milliard} de doses ont été administrées en Europe de
  \textbf{2021 à 2024}.
\end{itemize}

Sur le graphe d'évolution du nombre de personnes ayant reçu au moins une
dose de vaccin, nous observons une croissance négligeable de personnes
vaccinées début 2021. Cette tendance devient progressivement importante
vers la fin du premier semestre 2021, pour atteindre un pic moyen au
début du premier trimestre 2022.

Après cette période, elle est restée constante pour la \textbf{France,
l'Allemagne, l'Italie} et le \textbf{Royaume-Uni}.\\
Le nombre de personnes vaccinées en \textbf{Russie} a évolué jusqu'au
début du premier trimestre 2023, puis est resté constant jusqu'en 2024.

\hypertarget{observation-de-luxe9volution-des-cas-et-de-lindice-de-rigueur-selons-les-5-pays-europeennes-les-plus-touchuxe9s}{%
\subsubsection{Observation de l'Évolution des cas et de l'indice de
rigueur selons les 5 pays europeennes les plus
touchés}\label{observation-de-luxe9volution-des-cas-et-de-lindice-de-rigueur-selons-les-5-pays-europeennes-les-plus-touchuxe9s}}

\hypertarget{interpretation-2}{%
\paragraph{Interpretation :}\label{interpretation-2}}

On constate que la gestion de la pandémie de COVID-19 et l'application
des mesures sanitaires ont différé selon les pays européens étudiés. La
France, l'Italie, l'Allemagne et le Royaume-Uni ont tous réagi de façon
marquée lors des pics de contamination, en adoptant des restrictions
très strictes, avec des indices de rigueur pouvant atteindre 88 au
Royaume-Uni, surtout durant les années 2020 et 2021. À l'opposé, la
Russie a adopté une attitude plus souple, son indice de rigueur
dépassant rarement 60, même lors de vagues importantes, comme au début
de 2022. En Russie, il est aussi fréquent de voir les restrictions
maintenues à un niveau modéré pendant les pics, puis rapidement
assouplies, parfois avant même que la situation ne s'améliore
réellement. De manière générale, à partir de la mi-2022, tous ces pays
ont progressivement levé les mesures, malgré quelques hausses du nombre
de cas, ce qui reflète un changement de stratégie reposant davantage sur
la vaccination, la gestion des hôpitaux et l'acceptation par la
population. Cette analyse montre donc que chaque pays a adopté une
approche différente pour faire face à la pandémie, malgré un contexte
mondial~similaire.

\hypertarget{analyse-comparative}{%
\subsubsection{Analyse comparative}\label{analyse-comparative}}

Le suivi de l'évolution de la COVID-19 dans les cinq pays européens les
plus touchés -- France, Allemagne, Italie, Russie et Royaume-Uni --
montre des dynamiques variées en termes de nombre de cas et de
mortalité, influencées par la progression des campagnes de vaccination.
En début de pandémie, chacun de ces pays a enregistré des pics
importants de contamination, avec des tendances similaires jusqu'à fin
2021, où la France et l'Italie affichent des hausses rapides, culminant
respectivement à environ 350 000 et 180 000 cas quotidiens début 2022,
tandis que l'Allemagne et le Royaume-Uni connaissent des pics proches de
200 000 cas par jour. La Russie, quant à elle, présente des pics plus
étalés et moins élevés, avec un maximum autour de 150 000 cas par jour.
Ces variations s'expliquent notamment par des différences dans la
gestion de la crise et les capacités de dépistage.

Avec le déploiement massif de la vaccination dès 2021, une stabilisation
puis une baisse des nouveaux cas s'observent progressivement dans la
majorité des pays étudiés. En France, Allemagne, Italie et Royaume-Uni,
la couverture vaccinale atteint un plateau début 2022, période à partir
de laquelle la courbe des contaminations amorce une diminution notable,
signe d'une efficacité du vaccin à freiner la transmission. La Russie
diffère légèrement, avec une montée de la vaccination plus tardive,
jusqu'au début de 2023, ce qui coïncide avec une baisse plus tardive des
cas. La vaccination semble donc avoir joué un rôle clé non seulement en
limitant le nombre d'infections, mais aussi en atténuant l'intensité des
vagues successives.

Cette tendance se retrouve dans l'évolution de la mortalité. Les cinq
pays ont connu des vagues majeures de décès entre fin 2020 et début
2021, avant la généralisation de la vaccination, avec des pics dépassant
parfois les 1 000 morts par jour, notamment en Russie et au Royaume-Uni
où les chiffres sont particulièrement élevés. La France et l'Italie ont
suivi des trajectoires proches, marquées par une forte mortalité au
début de la crise puis une diminution progressive des décès. L'Allemagne
se distingue par des pics moins intenses mais plus prolongés dans le
temps, certains s'étalant jusqu'en 2023. En Russie, la mortalité élevée
persiste plus longtemps, reflétant une gestion plus complexe de la
pandémie. Le Royaume-Uni a connu les plus hauts sommets, avec près de 1
400 décès quotidiens en janvier 2021, avant d'observer une chute rapide
de la mortalité à partir de 2022. Globalement, la baisse significative
des décès dans tous ces pays à partir de 2022 correspond à
l'augmentation de la couverture vaccinale, ainsi qu'à l'adoption de
mesures sanitaires adaptées.

Ainsi, l'analyse conjointe des cas et des décès met en lumière l'effet
positif des campagnes de vaccination sur le contrôle de la pandémie.
Alors que les vagues de contamination ont pu être fortes et fréquentes
jusqu'au début de 2022, la vaccination a permis de limiter la gravité
des cas, conduisant à une réduction progressive et durable du nombre de
morts, même si la dynamique exacte varie selon le contexte national et
la stratégie sanitaire de chaque pays.

\hypertarget{section}{%
\subsection{}\label{section}}

\end{document}
